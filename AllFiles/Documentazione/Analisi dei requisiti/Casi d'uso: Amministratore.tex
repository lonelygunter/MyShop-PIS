\chapter{Amministratore}
		
\section{Creazione di un articolo}
\begin{itemize}
    \item \textbf{attore}: Amministratore
    \item \textbf{precondizione}: L’amministratore deve effettuare il Login
    
    \begin{enumerate}
        \item L’amministratore accede alla menu bar
        \label{itemCr1}
        \item L’amministratore seleziona la voce “articoli”
        \item Il sistema apre l’elenco degli articoli
        \item L’amministratore clicca sul pulsante “crea”
        \item Il sistema richiede i dati dell’articolo
    
		\begin{itemize}
        	\item PRODOTTO / COMPOSIZIONE DI PRODOTTI:
        	
			\begin{enumerate}
				\item nome
				\item immagini rappresentative
				\item descrizione max 255 char
				\item produttore
				\item costo
				\item collocazione nel magazzino self-service (corsia,scaffale)
				\item categoria
				\item sottocategoria
				\item uno o più prodotti al suo interno
			\end{enumerate}
		\end{itemize}

    
		\begin{itemize}
			\item SERVIZIO:
			
			\begin{enumerate}
				\item nome
    			\item immagini rappresentative
    			\item descrizione max 255 char
    			\item fornitore
    			\item costo
    			\item categoria
			\end{enumerate}
		\end{itemize}
    
    	\item L’amministratore inserisce i dati dell’articolo
    	\item L’amministratore clicca sul pulsante “crea”
    	\label{itemCr2}
    	\item Il sistema verifica i dati dell’articolo
    	\label{itemCr3}
    	\item Il sistema chiede la conferma della creazione
    	\item L’amministratore clicca su “conferma”
    	\item Il sistema memorizza i dati dell’articolo
    \end{enumerate}

    \item \textbf{estensioni}:
    \begin{enumerate}
		\item[\ref{itemCr1}a.] L’amministratore è appena entrato nel software
		\item Il sistema gli apre direttamente la schermata degli articoli
	\end{enumerate}
        
	\begin{enumerate}
		\item[\ref{itemCr2}a.] L’articolo inserito già esiste
		\item Il sistema chiede il reinserimento dei dati
	\end{enumerate}
        
	\begin{enumerate}
		\item[\ref{itemCr3}a.] L’amministratore non ha inserito dei dati
		\item Il sistema chiede l’inserimento dei dati mancanti
	\end{enumerate}	
            
    \item \textbf{postcondizione}: L’amministratore ha inserito un nuovo articolo nel catalogo
\end{itemize}
		

\section{Modifica di un articolo}
\begin{itemize}
	\item \textbf{attore}: Amministratore
	\item \textbf{precondizione}: L’amministratore deve effettuare il Login

	\begin{enumerate}
		\item L’amministratore accede alla menu bar
		\label{itemMod1}
		\item L’amministratore seleziona la voce “articoli”
		\item Il sistema apre l’elenco degli articoli
		\item L’amministratore clicca col tasto destro sull’articolo
		\item L’amministratore, dal menu a tendina, sceglie l’opzione “modifica”
		\item Il sistema mostra i dati dell’articolo
		\item L’amministratore modifica i dati dell’articolo
		\item L’amministratore clicca sul pulsante “modifica”
		\label{itemMod2}
		\item Il sistema verifica i dati dell’articolo
		\item Il sistema chiede la conferma della modifica
		\item L’amministratore clicca su “conferma”
		\item Il sistema memorizza i dati dell’articolo
	\end{enumerate}

	\item \textbf{estensioni}:
	\begin{enumerate}
		\item[\ref{itemMod1}a.] L’amministratore è appena entrato nel software
		\item Il sistema gli apre direttamente la schermata degli articoli
	\end{enumerate}

	\begin{enumerate}
		\item[\ref{itemMod2}a.] L’articolo inserito già esiste
		\item Il sistema chiede il reinserimento dei dati
	\end{enumerate}

	\item \textbf{postcondizione}: L’amministratore ha modificato un articolo nel catalogo
\end{itemize}


\section{Cancellazione di un articolo}
\begin{itemize}
	\item \textbf{attore}: Amministratore
	\item \textbf{precondizione}: L’amministratore deve effettuare il login
	
	\begin{enumerate}
		\item L’amministratore accede alla menu bar
		\label{itemDel1}
		\item L’amministratore seleziona la voce “articoli”
		\item Il sistema apre l’elenco degli articoli
		\item L’amministratore clicca col tasto destro sull’articolo
		\item L’amministratore, dal menu a tendina, sceglie l’opzione “cancella”
		\item Il sistema chiede la conferma della cancellazione
		\item L’amministratore clicca su “conferma”
	\end{enumerate}

	\item \textbf{estensioni}:
	\begin{enumerate}
		\item[\ref{itemDel1}a.] L’amministratore è appena entrato nel software
		\item Il sistema gli apre direttamente la schermata degli articoli
	\end{enumerate}

	\item \textbf{postcondizione}: L’amministratore ha eliminato un articolo
\end{itemize}

		
\section{Creazione di un produttore}
\begin{itemize}
	\item \textbf{attore}: Amministratore
	\item \textbf{precondizione}: L’amministratore deve effettuare il login

	\begin{enumerate}
		\item L’amministratore accede alla menu bar
		\item L’amministratore seleziona la voce “produttori”
		\item Il sistema apre l’elenco dei produttori
		\item L’amministratore clicca sul pulsante “crea”
		\item Il sistema richiede i dati del produttore
		\begin{enumerate}
			\item nome, email, telefono, sito web, città, nazione
		\end{enumerate}
		\item L’amministratore inserisce i dati del produttore
		\item L’amministratore clicca sul pulsante “crea”
		\label{prodCr1}
		\item Il sistema verifica i dati del produttore
		\item Il sistema chiede la conferma della creazione
		\item L’amministratore clicca su “conferma”
		\item Il sistema memorizza i dati del produttore
	\end{enumerate}

	\item \textbf{estensioni}:
	\begin{enumerate}
		\item[\ref{prodCr1}a.] I dati inseriti già esistono
		\item Il sistema chiede il reinserimento dei dati
	\end{enumerate}

	\item \textbf{postcondizione}: L’amministratore ha inserito un nuovo produttore
\end{itemize}

		
\section{Modifica di un produttore}
\begin{itemize}
	\item \textbf{attore}: Amministratore
	\item \textbf{precondizione}: L’amministratore deve effettuare il Login

	\begin{enumerate}
		\item L’amministratore accede alla menu bar
		\item L’amministratore seleziona la voce “produttori”
		\item Il sistema apre l’elenco dei produttori
		\item L’amministratore clicca col tasto destro sul produttore
		\item L’amministratore, dal menu a tendina, sceglie l’opzione “modifica”
		\item Il sistema mostra i dati del produttore
		\item L’amministratore modifica i dati del produttore
		\item L’amministratore clicca sul pulsante “modifica”
		\label{prodMod1}
		\item Il sistema verifica i dati del produttore
		\item Il sistema chiede la conferma della modifica
		\item L’amministratore clicca su “conferma”
		\item Il sistema memorizza i dati del produttore
	\end{enumerate}

	\item \textbf{estensioni}:
	\begin{enumerate}
		\item[\ref{prodMod1}a.] I dati inseriti già esiste
		\item Il sistema chiede il reinserimento dei dati
	\end{enumerate}

	\item \textbf{postcondizione}: L’amministratore ha modificato un produttore
\end{itemize}
		

\section{Cancellazione di un produttore}
\begin{itemize}
	\item \textbf{attore}: Amministratore
	\item \textbf{precondizione}: L’amministratore deve effettuare il login
	
	\begin{enumerate}
		\item L’amministratore accede alla menu bar
		\item L’amministratore seleziona la voce “produttori”
		\item Il sistema apre l’elenco dei produttore
		\item L’amministratore clicca col tasto destro sul produttore
		\item L’amministratore, dal menu a tendina, sceglie l’opzione “cancella”
		\item Il sistema chiede la conferma della cancellazione
		\item L’amministratore clicca su “conferma”
	\end{enumerate}

	\item \textbf{postcondizione}: L’amministratore ha eliminato un produttore
\end{itemize}


\section{Creazione di un fornitore}
\begin{itemize}
	\item \textbf{attore}: Amministratore
	\item \textbf{precondizione}: L’amministratore deve effettuare il login
	
	\begin{enumerate}
		\item L’amministratore accede alla menu bar
		\item L’amministratore seleziona la voce “fornitori”
		\item Il sistema apre l’elenco dei fornitori
		\item L’amministratore clicca sul pulsante “crea”
		\item Il sistema richiede i dati del fornitore
		\begin{enumerate}
			\item nome, email, telefono, sito web, città, nazione
		\end{enumerate}
		\item L’amministratore inserisce i dati del fornitore
		\item L’amministratore clicca sul pulsante “crea”
		\label{forCr1}
		\item Il sistema verifica i dati del fornitore
		\item Il sistema chiede la conferma della creazione
		\item L’amministratore clicca su “conferma”
		\item Il sistema memorizza i dati del fornitore
	\end{enumerate}

	\item \textbf{estensioni}:
	\begin{enumerate}
		\item[\ref{forCr1}a.] I dati inseriti già esistono 
		\item Il sistema chiede il reinserimento dei dati
	\end{enumerate}

	\item \textbf{postcondizione}: L’amministratore ha inserito un nuovo fornitore
\end{itemize}
		

\section{Modifica di un fornitore}
\begin{itemize}
	\item \textbf{attore}: Amministratore
	\item \textbf{precondizione}: L’amministratore deve effettuare il Login
	
	\begin{enumerate}
		\item L’amministratore accede alla menu bar
		\item L’amministratore seleziona la voce “fornitori”
		\item Il sistema apre l’elenco dei fornitori
		\item L’amministratore clicca col tasto destro sul fornitore
		\item L’amministratore, dal menu a tendina, sceglie l’opzione “modifica”
		\item Il sistema mostra i dati sul fornitore
		\item L’amministratore modifica i dati sul fornitore
		\item L’amministratore clicca sul pulsante “modifica”
		\label{forMod1}
		\item Il sistema verifica i dati del fornitore
		\item Il sistema chiede la conferma della modifica
		\item L’amministratore clicca su “conferma”
		\item Il sistema memorizza i dati del fornitore
	\end{enumerate}

	\item \textbf{estensioni}:
	\begin{enumerate}
		\item[\ref{forMod1}a.] I dati inseriti già esistono
		\item Il sistema chiede il reinserimento dei dati
	\end{enumerate}

	\item \textbf{postcondizione}: L’amministratore ha modificato un fornitore
\end{itemize}
		

\section{Cancellazione di un fornitore}
\begin{itemize}
	\item \textbf{attore}: Amministratore
	\item \textbf{precondizione}: L’amministratore deve effettuare il login

	\begin{enumerate}
		\item L’amministratore accede alla menu bar
		\item L’amministratore seleziona la voce “fornitori”
		\item Il sistema apre l’elenco dei fornitori
		\item L’amministratore clicca col tasto destro sul fornitore
		\item L’amministratore, dal menu a tendina, sceglie l’opzione “cancella”
		\item Il sistema chiede la conferma della cancellazione
		\item L’amministratore clicca su “conferma”
	\end{enumerate}

	\item \textbf{postcondizione}: L’amministratore ha eliminato un fornitore
\end{itemize}


\section{Creazione di una categoria}
\begin{itemize}
	\item \textbf{attore}: Amministratore
	\item \textbf{precondizione}: L’amministratore deve effettuare il login
	
	\begin{enumerate}
		\item L’amministratore accede alla menu bar
		\item L’amministratore seleziona la voce “categorie”
		\item Il sistema apre l’elenco delle categorie
		\item L’amministratore clicca sul pulsante “crea”
		\item Il sistema richiede i dati della categoria
		\item L’amministratore inserisce i dati della categoria
		\item L’amministratore clicca sul pulsante “crea”
		\label{catCr1}
		\item Il sistema verifica i dati della categoria
		\item Il sistema chiede la conferma della creazione
		\item L’amministratore clicca su “conferma”
		\item Il sistema memorizza i dati della categoria
	\end{enumerate}

	\item \textbf{estensioni}:
	\begin{enumerate}
		\item[\ref{catCr1}a.] I dati inseriti già esistono 
		\item Il sistema chiede il reinserimento dei dati
	\end{enumerate}

	\item \textbf{postcondizione}: L’amministratore ha inserito una nuova categoria
\end{itemize}


\section{Modifica di una categoria}
\begin{itemize}
	\item \textbf{attore}: Amministratore
	\item \textbf{precondizione}: L’amministratore deve effettuare il Login
	
	\begin{enumerate}
		\item L’amministratore accede alla menu bar
		\item L’amministratore seleziona la voce “categorie”
		\item Il sistema apre l’elenco delle categorie
		\item L’amministratore clicca col tasto destro sulla categoria
		\item L’amministratore, dal menu a tendina, sceglie l’opzione “modifica”
		\item Il sistema mostra i dati sulla categoria
		\item L’amministratore modifica i dati sulla categoria
		\item L’amministratore clicca sul pulsante “modifica”
		\label{catMod1}
		\item Il sistema verifica i dati della categoria
		\item Il sistema chiede la conferma della modifica
		\item L’amministratore clicca su “conferma”
		\item Il sistema memorizza i dati della categoria
	\end{enumerate}

	\item \textbf{estensioni}:
	\begin{enumerate}
		\item[\ref{catMod1}a.] I dati inseriti già esistono
		\item Il sistema chiede il reinserimento dei dati
	\end{enumerate}

	\item \textbf{postcondizione}: L’amministratore ha modificato una categoria
\end{itemize}


\section{Cancellazione di una categoria}
\begin{itemize}
	\item \textbf{attore}: Amministratore
	\item \textbf{precondizione}: L’amministratore deve effettuare il login
	
	\begin{enumerate}
		\item L’amministratore accede alla menu bar
		\item L’amministratore seleziona la voce “categorie”
		\item Il sistema apre l’elenco delle categorie
		\item L’amministratore clicca col tasto destro sulla categoria
		\item L’amministratore, dal menu a tendina, sceglie l’opzione “cancella”
		\item Il sistema chiede la conferma della cancellazione
		\item L’amministratore clicca su “conferma”
	\end{enumerate}

	\item \textbf{postcondizione}: L’amministratore ha eliminato una categoria
\end{itemize}


\section{Creazione di una sottocategoria}
\begin{itemize}
	\item \textbf{attore}: Amministratore
	\item \textbf{precondizione}: L’amministratore deve effettuare il login

	\begin{enumerate}
		\item L’amministratore accede alla menu bar
		\item L’amministratore seleziona la voce “categorie”
		\item Il sistema apre l’elenco delle categorie
		\item L’amministratore clicca col tasto destro sull’articolo
		\item L’amministratore, dal menu a tendina, sceglie l’opzione “modifica”
		\item Il sistema mostra i dati della categoria
		\item L’amministratore clicca sul pulsante “+”
		\label{scatCr1}
		\item Il sistema richiede il nome della sottocategoria
		\item L’amministratore inserisce i dati della sottocategoria
		\item L’amministratore clicca sul pulsante “crea”
		\item Il sistema verifica i dati della sottocategoria
		\item Il sistema chiede la conferma della creazione
		\item L’amministratore clicca su “conferma”
		\item Il sistema memorizza i dati della sottocategoria
	\end{enumerate}

	\item \textbf{estensioni}:
	\begin{enumerate}
		\item[\ref{scatCr1}a.] La sottocategoria già esiste
		\item Il sistema chiede il reinserimento del nome
	\end{enumerate}

	\item \textbf{postcondizione}: L’amministratore ha inserito una nuova sottocategoria	
\end{itemize}


\section{Creazione punto vendita}
\begin{itemize}
	\item \textbf{attore}: Amministratore
	\item \textbf{precondizione}: L’amministratore deve effettuare il login
	
	\begin{enumerate}
		\item L’amministratore accede alla menu bar
		\item L’amministratore seleziona la voce “punti vendita”
		\item Il sistema apre l’elenco dei punti vendita
		\item L’amministratore clicca sul pulsante “crea”
		\item Il sistema richiede i dati del punto vendita
		\begin{enumerate}
			\item nome, categorie di articoli venduti, manager, posizione ...
		\end{enumerate}
		\item L’amministratore inserisce i dati del punto vendita
		\item L’amministratore clicca sul pulsante “crea”
		\item Il sistema verifica i dati del punto vendita
		\label{storeCr1}
		\item Il sistema chiede la conferma del punto vendita
		\item L’amministratore clicca su “conferma”
		\item Il sistema memorizza i dati del punto vendita
	\end{enumerate}

	\item \textbf{estensioni}:
	\begin{enumerate}
		\item[\ref{storeCr1}a.] I dati inseriti già esistono 
		\item Il sistema chiede il reinserimento dei dati
	\end{enumerate}

	\item \textbf{postcondizione}: L’amministratore ha inserito un nuovo punto vendita
\end{itemize}


\section{Modifica di un punto vendita}
\begin{itemize}
	\item \textbf{attore}: Amministratore
	\item \textbf{precondizione}: L’amministratore deve effettuare il Login
	
	\begin{enumerate}
		\item L’amministratore accede alla menu bar
		\item L’amministratore seleziona la voce “punti vendita”
		\item Il sistema apre l’elenco dei punti vendita
		\item L’amministratore clicca col tasto destro sul punto  vendita
		\item L’amministratore, dal menu a tendina, sceglie l’opzione “modifica”
		\item Il sistema mostra i dati del punto vendita
		\item L’amministratore modifica i dati sul punto vendita
		\item L’amministratore clicca sul pulsante “modifica”
		\label{storeMod1}
		\item Il sistema verifica i dati del punto vendita
		\item Il sistema chiede la conferma del punto vendita
		\item L’amministratore clicca su “conferma”
		\item Il sistema memorizza i dati del punto vendita
	\end{enumerate}

	\item \textbf{estensioni}:
	\begin{enumerate}
		\item[\ref{storeMod1}a.] I dati inseriti già esistono
		\item Il sistema chiede il reinserimento dei dati
	\end{enumerate}

	\item \textbf{postcondizione}: L’amministratore ha modificato un punto vendita
\end{itemize}


\section{Cancellazione di un punto vendita}
\begin{itemize}
	\item \textbf{attore}: Amministratore
	\item \textbf{precondizione}: L’amministratore deve effettuare il login
	
	\begin{enumerate}
		\item L’amministratore accede alla menu bar
		\item L’amministratore seleziona la voce “punti vendita”
		\item Il sistema apre l’elenco dei punti vendita
		\item L’amministratore clicca col tasto destro sul punto vendita
		\item L’amministratore, dal menu a tendina, sceglie l’opzione “cancella”
		\item Il sistema chiede la conferma della cancellazione
		\item L’amministratore clicca su “conferma”
	\end{enumerate}

	\item \textbf{postcondizione}: L’amministratore ha eliminato un punto vendita
\end{itemize}


\section{Creazione di un Manager}
\begin{itemize}
	\item \textbf{attore}: Amministratore
	\item \textbf{precondizione}:  L’amministratore deve effettuare il login
	
	\begin{enumerate}
		\item L’amministratore accede alla menu bar
		\item L’amministratore seleziona la voce “manager”
		\item Il sistema apre l’elenco dei Manager
		\item L’amministratore clicca sul pulsante “crea”
		\item Il sistema richiede i dati del Manager
		\item L’amministratore inserisce i dati del Manager
		\item L’amministratore clicca sul pulsante “crea”
		\label{manCr1}
		\item Il sistema verifica i dati del Manager
		\item Il sistema chiede la conferma del Manager
		\item L’amministratore clicca su “conferma”
		\item Il sistema memorizza i dati del Manager
	\end{enumerate}

	\item \textbf{estensioni}:
	\begin{enumerate}
		\item[\ref{manCr1}a.] I dati inseriti già esistono 
		\item Il sistema chiede il reinserimento dei dati
	\end{enumerate}

	\item \textbf{postcondizione}: L’amministratore ha inserito un nuovo Manager
\end{itemize}

		
\section{Modifica di un Manager}
\begin{itemize}
	\item \textbf{attore}: Amministratore
	\item \textbf{precondizione}: L’amministratore deve effettuare il Login
	
	\begin{enumerate}
		\item L’amministratore accede alla menu bar
		\item L’amministratore seleziona la voce “Manager”
		\item Il sistema apre l’elenco dei Manager
		\item L’amministratore clicca col tasto destro sul Manager
		\item L’amministratore, dal menu a tendina, sceglie l’opzione “modifica”
		\item Il sistema mostra i dati del Manager
		\item L’amministratore modifica i dati sul Manager
		\item L’amministratore clicca sul pulsante “modifica”
		\label{manMod1}
		\item Il sistema verifica i dati del Manager
		\item Il sistema chiede la conferma del Manager
		\item L’amministratore clicca su “conferma”
		\item Il sistema memorizza i dati del Manager
	\end{enumerate}

	\item \textbf{estensioni}:
	\begin{enumerate}
		\item[\ref{manMod1}a.] I dati inseriti già esistono
		\item Il sistema chiede il reinserimento dei dati
	\end{enumerate}

	\item \textbf{postcondizione}: L’amministratore ha modificato un Manager
\end{itemize}

		
\section{Cancellazione di un Manager}
\begin{itemize}
	\item \textbf{attore}: Amministratore
	\item \textbf{precondizione}: L’amministratore deve effettuare il login
	
	\begin{enumerate}
		\item L’amministratore accede alla menu bar
		\item L’amministratore seleziona la voce “Manager”
		\item Il sistema apre l’elenco dei Manager
		\item L’amministratore clicca col tasto destro sul Manager
		\item L’amministratore, dal menu a tendina, sceglie l’opzione “cancella”
		\item Il sistema chiede la conferma della cancellazione
		\item L’amministratore clicca su “conferma”
	\end{enumerate}

	\item \textbf{postcondizione}: L’amministratore ha eliminato un Manager
\end{itemize}