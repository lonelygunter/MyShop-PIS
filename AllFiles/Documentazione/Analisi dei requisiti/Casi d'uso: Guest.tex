\chapter{Utente guest}


\section{Registrazione}
\begin{itemize}
    \item \textbf{attore}: Utente Guest
    \item \textbf{precondizioni}: n/a
    
    \begin{enumerate}
        \item L’utente accede all’area di registrazione
        \item L’utente inserisce i suoi dati
        \label{reg1}
        \begin{enumerate}
            \item nome, cognome, e-mail, telefono, età, residenza, professione
        \end{enumerate}
        \item Il sistema verifica i dati non siano già esistenti
        \item Il sistema gli memorizza
        \item Il sistema torna alla Home
    \end{enumerate}

    \item \textbf{estensioni}:
    \begin{enumerate}
        \item[\ref{reg1}a.] I dati sono già esistenti
        \item Il sistema avvisa l’utente che i dati sono già esistenti
    \end{enumerate}

    \item \textbf{postcondizione}: L’utente guest si è registrato
\end{itemize}
