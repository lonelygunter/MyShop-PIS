\chapter{Utente acquirente}


\section{Prenotazione di un articolo}
\begin{itemize}
    \item \textbf{attore}: Utente acquirente
    \item \textbf{precondizione}: L’utente acquirente effettua il login

    \begin{enumerate}
        \item L’utente acquirente accede alla Home
        \item L’utente acquirente cerca, dall’apposita barra, l’articolo desiderato
        \item L’utente acquirente clicca sull’articolo interessato
        \item L’utente acquirente clicca sul pulsante “prenota subito”
        \item Il sistema chiede la conferma della prenotazione
        \label{itemPren1}
        \item L'utente acquirente clicca su “conferma”
        \item Il sistema prenota l’articolo
    \end{enumerate}

    \item \textbf{estensioni}:
    \begin{enumerate}
        \item[\ref{itemPren1}a.] La prenotazione non può avvenire
        \item Il sistema avvisa l’utente acquirente che non è possibile prenotare l’articolo desiderato
    \end{enumerate}

    \item \textbf{postcondizione}: L’utente acquirente ha prenotato l’articolo desiderato
\end{itemize}


\section{Creazione di una lista d’acquisto}
\begin{itemize}
    \item \textbf{attore}: Utente acquirente
    \item \textbf{precondizione}: L’utente acquirente deve effettua il login

    \begin{enumerate}
        \item L’utente acquirente accede alla Home
        \item L’utente acquirente clicca sul pulsante “liste”
        \item Il sistema apre l’elenco delle liste
        \item L’utente acquirente clicca sul pulsante “crea”
        \item Il sistema richiede il nome della lista 
        \item L’utente acquirente inserisce i dati della lista
        \item L’utente acquirente clicca sul pulsante “conferma”
        \item Il sistema verifica i dati della lista
        \label{listCr1}
        \item Il sistema memorizza i dati della lista
    \end{enumerate}

    \item \textbf{estensioni}:
    \begin{enumerate}
        \item[\ref{listCr1}a.] La lista già esiste
        \item Il sistema notifica che la lista già esiste
    \end{enumerate}

    \item \textbf{postcondizione}: L’utente acquirente ha creato una nuova lista
\end{itemize}
		

\section{Modifica di una lista d’acquisto}
\begin{itemize}
    \item \textbf{attore}: Utente acquirente
    \item \textbf{precondizione}: L’utente acquirente deve effettua il login
    
    \begin{enumerate}
        \item L’utente acquirente accede alla Home
        \item L’utente acquirente clicca sul pulsante “liste”
        \item Il sistema apre l’elenco delle liste
        \item L’utente acquirente clicca sulla lista interessata
        \item Il sistema apre la lista 
        \item L’utente acquirente elimina un articolo tramite il pulsante “-”
        \item Il sistema chiede la conferma dell’eliminazione
        \item L’utente acquirente clicca sul pulsante “conferma”
        \item Il sistema cancella l’articolo dalla lista
    \end{enumerate}
    
    \item \textbf{postcondizione}: L’utente acquirente ha eliminato un articolo dalla lista
\end{itemize}


\section{Cancellazione di una lista d’acquisto}
\begin{itemize}
    \item \textbf{attore}: Utente acquirente
    \item \textbf{precondizione}: L’utente acquirente deve effettua il login
    
    \begin{enumerate}
        \item L’utente acquirente accede alla Home
        \item L’utente acquirente clicca sul pulsante “liste”
        \item Il sistema apre l’elenco delle liste
        \item L’utente acquirente elimina una lista tramite il pulsante “-”
        \item Il sistema chiede la conferma dell’eliminazione
        \item L’utente acquirente clicca sul pulsante “conferma”
        \item Il sistema cancella l’articolo dalla lista
    \end{enumerate}

    \item \textbf{postcondizione}: L’utente acquirente ha eliminato una lista
\end{itemize}
		

\section{Aggiungere un articolo in una lista d’acquisto}
\begin{itemize}
    \item \textbf{attore}: Utente acquirente
    \item \textbf{precondizione}: L’utente acquirente deve effettua il login

    \begin{enumerate}
        \item L’utente acquirente accede alla Home
        \item L’utente acquirente clicca sul pulsante “+” presente dell’anteprima dell’articolo
        \item Il sistema apre l’elenco delle liste già create
        \item L’utente acquirente sceglie la lista alla quale aggiungere l’articolo
        \label{listAdd1}
        \item Il sistema memorizza l’articolo nella lista
    \end{enumerate}

    \item \textbf{estensioni}:
    \begin{enumerate}
        \item[\ref{listAdd1}a.] La lista non esiste
        \item L’utente acquirente crea una nuova lista
    \end{enumerate}

    \item \textbf{postcondizione}:  L’utente acquirente ha creata la sua lista d’acquisto
\end{itemize}


\section{Creazione di un pdf della lista d’acquisto}
\begin{itemize}
    \item \textbf{attore}: Utente acquirente
    \item \textbf{precondizione}: L’utente acquirente deve effettuare il login
    
    \begin{enumerate}
        \item L’utente acquirente accede alla Home
        \item L’utente acquirente clicca sul pulsante “liste”
        \item Il sistema apre l’elenco delle liste
        \item L’utente acquirente clicca col tasto destro sulla lista interessata
        \item L’utente acquirente, dal menu a tendina, sceglie l’opzione “invia pdf”
        \item Il sistema crea un pdf della lista d’acquisto
        \item Il sistema spedisce via e-mail il pdf
        \item L’utente acquirente riceve l’email
    \end{enumerate}

    \item \textbf{postcondizione}: L’utente acquirente ha ricevuto il pdf della sua lista d’acquisto
\end{itemize}


\section{Rilascio di feedback}
\begin{itemize}
    \item \textbf{attore}: Utente acquirente
    \item \textbf{precondizione}: L’utente acquirente deve effettuare il login 

    \begin{enumerate}
        \item L’utente acquirente accede alla Home
        \item L’utente acquirente cerca, dall’apposita barra, l’articolo desiderato
        \item L’utente acquirente clicca col tasto sinistro l’articolo interessato
        \item L’utente acquirente scorre fino ad arrivare alla sezione feedback
        \item L’utente acquirente commenta nell’apposito spazio
    \end{enumerate}

    \item \textbf{postcondizione}: L’utente acquirente ha rilasciato un feedback
\end{itemize}
